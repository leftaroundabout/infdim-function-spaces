%% For double-blind review submission, w/o CCS and ACM Reference (max submission space)
\documentclass[sigplan,review,anonymous]{acmart}\settopmatter{printfolios=true,printccs=false,printacmref=false}
%% For double-blind review submission, w/ CCS and ACM Reference
%\documentclass[sigplan,review,anonymous]{acmart}\settopmatter{printfolios=true}
%% For single-blind review submission, w/o CCS and ACM Reference (max submission space)
%\documentclass[sigplan,review]{acmart}\settopmatter{printfolios=true,printccs=false,printacmref=false}
%% For single-blind review submission, w/ CCS and ACM Reference
%\documentclass[sigplan,review]{acmart}\settopmatter{printfolios=true}
%% For final camera-ready submission, w/ required CCS and ACM Reference
%\documentclass[sigplan]{acmart}\settopmatter{}


%% Conference information
%% Supplied to authors by publisher for camera-ready submission;
%% use defaults for review submission.
\acmConference[ICFP'19]{ACM SIGPLAN International Conference on Functional Programming}{August 18--23, 2019}{Berlin, Germany}
\acmYear{2019}
\acmISBN{} % \acmISBN{978-x-xxxx-xxxx-x/YY/MM}
\acmDOI{} % \acmDOI{10.1145/nnnnnnn.nnnnnnn}
\startPage{1}

%% Copyright information
%% Supplied to authors (based on authors' rights management selection;
%% see authors.acm.org) by publisher for camera-ready submission;
%% use 'none' for review submission.
\setcopyright{none}
%\setcopyright{acmcopyright}
%\setcopyright{acmlicensed}
%\setcopyright{rightsretained}
%\copyrightyear{2018}           %% If different from \acmYear

%% Bibliography style
\bibliographystyle{ACM-Reference-Format}
%% Citation style
%\citestyle{acmauthoryear}  %% For author/year citations
%\citestyle{acmnumeric}     %% For numeric citations
%\setcitestyle{nosort}      %% With 'acmnumeric', to disable automatic
                            %% sorting of references within a single citation;
                            %% e.g., \cite{Smith99,Carpenter05,Baker12}
                            %% rendered as [14,5,2] rather than [2,5,14].
%\setcitesyle{nocompress}   %% With 'acmnumeric', to disable automatic
                            %% compression of sequential references within a
                            %% single citation;
                            %% e.g., \cite{Baker12,Baker14,Baker16}
                            %% rendered as [2,3,4] rather than [2-4].


%%%%%%%%%%%%%%%%%%%%%%%%%%%%%%%%%%%%%%%%%%%%%%%%%%%%%%%%%%%%%%%%%%%%%%
%% Note: Authors migrating a paper from traditional SIGPLAN
%% proceedings format to PACMPL format must update the
%% '\documentclass' and topmatter commands above; see
%% 'acmart-pacmpl-template.tex'.
%%%%%%%%%%%%%%%%%%%%%%%%%%%%%%%%%%%%%%%%%%%%%%%%%%%%%%%%%%%%%%%%%%%%%%


%% Some recommended packages.
\usepackage{booktabs}   %% For formal tables:
                        %% http://ctan.org/pkg/booktabs
\usepackage{subcaption} %% For complex figures with subfigures/subcaptions
                        %% http://ctan.org/pkg/subcaption

\usepackage{listings}
\lstset{ language=haskell%
       , columns=flexible }

\begin{document}

%% Title information
\title[Lazy Infinite Wavelet Expansion]
      {Lazy Evaluation in Infinite-Dimensional Function Spaces with Wavelet Basis}


%% Author information
%% Contents and number of authors suppressed with 'anonymous'.
%% Each author should be introduced by \author, followed by
%% \authornote (optional), \orcid (optional), \affiliation, and
%% \email.
%% An author may have multiple affiliations and/or emails; repeat the
%% appropriate command.
%% Many elements are not rendered, but should be provided for metadata
%% extraction tools.

%% Author with two affiliations and emails.
\author{Olivier Verdier}
\authornote{with author2 note}          %% \authornote is optional;
                                        %% can be repeated if necessary
\orcid{nnnn-nnnn-nnnn-nnnn}             %% \orcid is optional
\affiliation{
  \position{Position2a}
  \department{Department2a}             %% \department is recommended
  \institution{Institution2a}           %% \institution is required
  \streetaddress{Street2a Address2a}
  \city{City2a}
  \state{State2a}
  \postcode{Post-Code2a}
  \country{Country2a}                   %% \country is recommended
}
\email{first2.last2@inst2a.com}         %% \email is recommended
\affiliation{
  \position{Position2b}
  \department{Department2b}             %% \department is recommended
  \institution{Institution2b}           %% \institution is required
  \streetaddress{Street3b Address2b}
  \city{City2b}
  \state{State2b}
  \postcode{Post-Code2b}
  \country{Country2b}                   %% \country is recommended
}
\email{first2.last2@inst2b.org}         %% \email is recommended

\author{Justus Sagemüller}
\affiliation{
  \department{Faculty of Engineering and Science}
  \institution{Høgskulen på Vestlandet}
  \streetaddress{Inndalsveien 28}
  \city{Bergen}
  \state{Hordaland}
  \postcode{5020}
  \country{Norway}
}
\email{mailto_js@gmx.de}


%% Abstract
%% Note: \begin{abstract}...\end{abstract} environment must come
%% before \maketitle command
\begin{abstract}
What is in numerics represented with “vectors” in form of arrays of numbers are often conceptually continuous functions, i.e. elements of a function vector-space.
It would be desirable to express this explicitly with types in the numerical code. A naïve reason why this is normally not done is that the function spaces are infinite-dimensional, whereas the numerical code must run in finite time and memory.

We argue that this is not a hurdle: even in an infinite-dimensional space, the vectors \emph{can} in practice be stored in finite memory. What does require infinite data structures for doing linear algebra are actually just \emph{dual vectors}, also called \emph{distributions}.
Those happen to be equivalent to vectors in the finite-dimensional case, which is why the distinction is usually blurred; but as we show an explicit type-level distinction between functions and distributions makes sense and allows directly expressing useful concepts such as the Dirac distribution, which are problematic in the standard finite-resolution picture.

The example implementation uses a very simple local basis that corresponds to a Haar Wavelet transform.
\end{abstract}


%% 2012 ACM Computing Classification System (CSS) concepts
%% Generate at 'http://dl.acm.org/ccs/ccs.cfm'.
\begin{CCSXML}
<ccs2012>
<concept>
<concept_id>10011007.10011006.10011008</concept_id>
<concept_desc>Software and its engineering~General programming languages</concept_desc>
<concept_significance>500</concept_significance>
</concept>
<concept>
<concept_id>10003456.10003457.10003521.10003525</concept_id>
<concept_desc>Social and professional topics~History of programming languages</concept_desc>
<concept_significance>300</concept_significance>
</concept>
</ccs2012>
\end{CCSXML}

\ccsdesc[500]{Software and its engineering~General programming languages}
\ccsdesc[300]{Social and professional topics~History of programming languages}
%% End of generated code


%% Keywords
%% comma separated list
\keywords{keyword1, keyword2, keyword3}  %% \keywords are mandatory in final camera-ready submission


%% \maketitle
%% Note: \maketitle command must come after title commands, author
%% commands, abstract environment, Computing Classification System
%% environment and commands, and keywords command.
\maketitle


\section{Introduction}

Consider the unit interval $D^1 = [-1,1] \subset \mathbb{R}$. This paper discusses functions on that domain, but the methods have been selected with thought to generalisation to unbounded multidimensional domains in mind.

The set of functions $D^1 \to \mathbb{R}$ is a vector space, but it is not only infinite- but even uncountably-dimensional, which makes any storing of coefficients -- i.e., of discrete function values in the continuous domain -- impractical indeed.
Fortunately, any subset of $\mathbb{R}^{D^1}$ which is closed under scaling and addition is a subspace. Well-studied examples include
\begin{itemize}
 \item $\mathcal{C}^0(D^1)$: continuous functions. These can be characterised thus: to obtain any function value $f(x)$ with at least precision $\varepsilon$, one can instead consider $f(\tilde{x})$ where $\tilde{x}$ needs to be merely \emph{close enough} to $x$, i.e. within a distance $\delta_{x,\varepsilon}$.
 \item $\mathcal{C}^1(D^1)$: continuously differentiable functions. % Discuss how this essentially means just there's a more efficient finite-basis-approximation opportunity
 \item $\mathcal{L}^2(D^1)$: square-integrable functions. % Integral definition
\end{itemize}
Continuity has a straightforward physical interpretation. It can be argued that a function on a non-discrete domain \emph{must} be continuous, at least almost everywhere,
if its values are supposed to be measurable: if it were not, then one would need to set up $x$ exactly before measuring $f(x)$ even approximately, but a physical setup can never be completely exact.
% Contrast with 𝓛² Hilbert space, why that is often preferred in practice, Riesz representation theorem, Nyquist reduction to finite sampling

\section{The space of PCM-sampled functions}
% Discuss standard approach – pre-select, fixed resolution finite-dim space – problem: risk of underestimating required reso or else wasting resources on unnecessary reso
It is possible to have an indeed infinite-dimensional vector-space type approximating $\mathcal{L}^2(D^1)$, with a PCM representation but the choice of a resolution deferred to runtime. The data structure itself is then simply an array (or list) of unspecified finite length $n$,
\begin{lstlisting}
newtype PCM_D¹ y = PCM_D¹ {
          getPCMSampling :: [y] }
\end{lstlisting}
where the \lstinline`y` values correspond to equally-spaced samples of the represented function, i.e.
\[
  \left[f(-1 + \tfrac2n\cdot i)\ \middle|\ i\leftarrow[\tfrac12\;..\;n-\tfrac12]\right].
\]
A harmless, but insightful difficulty with this approach is that vector addition does \emph{not} simply reduce to component-wise addition anymore.
% Demonstrate code&example for `instance AdditiveGroup PCM_D¹`, involving interpolation

A more fundamental problem with such an approach is that every vector needs to actually know its resolution. That is given in all purely forward-calculating applications; in that case vectors come in given and are processed by functions in the programming language, which do not store state.

Very many applications however exploit the possibility of linear algebra to store linear mappings in an efficient way as matrices, and to also perform inverse calculations (i.e. solving linear or locally-linearised equations).
% Cite Conal Elliott and/or other works that bring in cartesian closed categories; conclude that we want a category with, at the very least, identity mappings
% Discuss inability of PCM to express complete identity (infinite resolution in the contravariant part), and perhaps that it's inefficient even for a finite-dim cutoff (quadratic complexity of dense matrices).

\section{Abstract linear algebra}
% Primer on functional analysis, dual spaces, the `LinearSpace` type class

\section{Multiscale resolution}\label{mulScaleResoIntro}
A moral to take away from the usefulness of spaces like $\mathcal{L}^2$ is that the infinite dimensionality can be handled easily if for any concrete result, only finite-width \emph{integrals} over the function are required, as that averages out small-scale fluctuations. To actually benefit from this, it is necessary to be able to obtain the integral without actually evaluating all the small-scale structure. That is the idea behind multiscale or wavelet methods. These are often derived as some orthonormal basis of $\mathcal{L}^2$, but we can also give a construction more from first principles.

To directly enable the $\mathcal{O}(1)$ evaluation of large-scale integrals, one might consider the following representation of $D^1\to \mathrm{y}$ functions:
\begin{lstlisting}
data PreIntg_D¹ y = PreIntg
   { offset :: y
   , lSubstructure :: PreIntg_D¹ y
   , rSubstructure :: PreIntg_D¹ y
   }
\end{lstlisting}
The idea is to decompose a function into a constant offset (proportional to the integral $\int_{D^1}\!\mathrm{d}x\:f(x)$) plus finer-grained fluctuations in each half of the domain, which are in turn recursively represented by the same type.
\[
  f_{(y_0,f_\mathrm{l},f_\mathrm{r})}(x)
      = y_0 + \begin{cases}
                 f_\mathrm{l}(x_\mathrm{l}) & \text{if $x$ on left}
              \\ f_\mathrm{r}(x_\mathrm{r}) & \text{if $x$ on right}
              \end{cases}
\]
The above \lstinline`PreIntg_D¹` data structure is a tree which has always infinite size; this can be handled in a language like Haskell through lazy evaluation, however only if all that is ever requested from the function are integrals over finite-extend subintervals. Pointwise evaluation would recurse infinitely.

Always and everywhere going to infinite resolution is overzealous for a function type. It is not really an $D^1\to \mathrm{y}$ function if it \emph{cannot} be evaluated at any individual point in $D^1$. In practice, for any given real-world measured function, there will be only finitely many data points available anywhere, so in practice one will at sufficiently small scale eventually store \emph{only} the offset, and presume that any even smaller fluctuations are neglectable, which is warranted to happen for a continuous function.

This cutoff can be achieved by wrapping the substructure fields in \lstinline`Maybe` or, equivalenty, offering a dedicated constructor for the zero function, resulting in a conventional finite tree which could also be strictly evaluated (here enforced by the exclamation marks).
\begin{lstlisting}
data PreIntg_D¹ y
      = PreIntgZero
      | PreIntg !y !(PreIntg_D¹ y)
                   !(PreIntg_D¹ y)
\end{lstlisting}
Pointwise function evaluation is then readily implemented recursively
\begin{lstlisting}
evalPreIntg_D¹ :: AdditiveGroup y
     => PreIntg_D¹ y -> D¹ -> y
evalPreIntg_D¹ PreIntgZero _ = 0
evalPreIntg_D¹ (PreIntg y0 l r) x
   = y0 + if x < 0
           then evalPreIntg_D¹ l (2*x+1)
           else evalPreIntg_D¹ r (2*x-1)
\end{lstlisting}
Here, \lstinline`2*x+1` or \lstinline`2*x-1` “zoom in” onto the left or right half subinterval, depending on where \lstinline`x` resides.

\par
Awkward about \lstinline`data PreIntg_D¹` is that it contains redundant information: the offset already fixes what the integral over the complete function should be, but in fact there is nothing preventing the sub-interval functions from contributing their own part to the integral. This can be prevented if they are instead in a type that represents by construction integral-free functions. This means there can be no global offset, instead the highest-level structure is the offset \emph{difference} between the domain halves. This changes nothing about the data structure, just about its meaning:
\begin{lstlisting}
data HaarUnbiased y
     = HaarZero
     | HaarUnbiased !y !(HaarUnbiased y)
                       !(HaarUnbiased y)
\end{lstlisting}
Here, the \lstinline`y` value represents now basically the difference in offset between the left and right halves, or by our convention: the offset in the right half and simultaneously the negated offset in the left half (which must be the same, for the integral to come out as zero).
\begin{figure}
 \centering
 \includegraphics[width=\linewidth]{Haar-domDecompose.pdf}
 \caption{Example of how a function $f:D^1\to\mathbb{R}$ is decomposed into a constant offset, plus a step-function (Haar wavelet) for the offset-difference between left and right half, plus local fluctuations in each of the halves.}
 \label{haarDomDecompose}
\end{figure}
To still support functions with nonzero integral, one can simply add an absolute offset with a wrapper-type only at the top level:
\begin{lstlisting}
data Haar_D¹ dn y = Haar_D¹
    { global_offset :: !y
    , variation :: HaarUnbiased y }
\end{lstlisting}
The name “Haar” indicates that the basis functions of this data type (meaning those functions represented when exactly one of the fields of type ℝ in the \lstinline`HaarUnbiased ℝ` structure is 1, all other zero) are exactly the unnormalised Haar wavelets.

\section{Sampling}
Both with traditional orthonormal-basis methods and the domain-decomposition approach introduced in section \ref{mulScaleResoIntro}, the numerical representations are conceptually obtained from a function on the interval by integration. In case the function is already given by an analytic or other numerical expression, it may be possible to calculate this integral exactly, but more typically it will in real-world applications actually be calculated numerically itself, from a finite sample of discrete points. All such methods (which if fact consistently describe the integral) amount to some weighted average of the sample points:
\[
  \int_{D^1}\!\mathrm{d}x\:f(x) \approx \sum_i w_i\cdot f(x_i)
\]
with $x_i\in D^1$, $\sum_i w_i = 1$. For the choice of evaluation points and weights there are many different considerations to achieve good accuracy efficiently, this shall not be discussed here.

Crucially for our purposes, the calculation can be split up across the domain just like the recursive \lstinline`HaarUnbiased` data structure is:
\[
  \int_{D^1}\!\mathrm{d}x\:f(x)
    = \frac12\int_{D^1}\!\mathrm{d}x\:f(\tfrac{x-1}2)
    + \frac12\int_{D^1}\!\mathrm{d}x\:f(\tfrac{x+1}2)
\]
Observe that $\tfrac{x-1}2$ only evaluates on the left-, $\tfrac{x+1}2$ only on the right half of the domain.

This allows constructing the \lstinline`HaarUnbiased` tree in single pass with bottom-up propagation of the partial integrals, to obtain the offset estimates at each level without redundant computation. The choice of actually discrete numerical approximation only needs to be made at the smallest level; the simplest option is to only evaluate it at a single point in the middle and give that full weight (rectangular method). Thus the Haar representation can be obtained in $\mathcal{O}(n\cdot\log n)$ from a function on the interval:
\begin{lstlisting}
homsampleHaar_D¹ :: ( VectorSpace y
                    , Fractional (Scalar y) )
            => PowerOfTwo -> (D¹ -> y) -> Haar_D¹ y
homsampleHaar_D¹ (TwoToThe 0) f
   = Haar_D¹ (f 0) HaarZero
homsampleHaar_D¹ (TwoToThe i) f
   = case homsampleHaar_D¹ (TwoToThe $ i-1)
            <$> [ f . \x -> (x-1)/2
                , f . \x -> (x+1)/2 ] of
       [Haar_D¹ y0l sfl, Haar_D¹ y0r sfr]
        -> Haar_D¹ ((y0l+y0r)/2)
             $ HaarUnbiased ((y0r-y0l)/2) sfl sfr
\end{lstlisting}
This algorithm is in DSP called a \emph{fast wavelet transform}, which starts out with a PCM-sampled array instead of a function-to-be-sampled.
One advantage of our approach is that it is not really necessary to select one global a-priori maximum resolution (here the \lstinline`PowerOfTwo` parameter); instead, a heuristic can be added that refines the resolution \emph{locally} until the function is sufficiently well approximated.
Such adaptive resolution is known to boost performance in many real-world applications, such as physics/engineering simulations (finite elements or finite volumes methods, where it is called adaptive mesh refinement). It is also the main principle behind image compression formats which use quantization on a wavelet expansion.
The reason is that images or solutions to nonliner differential equations are often quite smooth in most of the domain, but include sharp edges / transients / shocks confined to a much smaller area.

\section{Dual space: functionals}
The dual space of a function space $X$ contains itself functions, but at least in principle on a different domain:
\begin{align*}
  X \subset& (D^1 \to \mathbb{R})
 \\
  X^\ast \subset& \bigl((D^1 \to \mathbb{R}) \to \mathbb{R}\bigr).
\end{align*}
I.e., $X$ contains higher-order functions, also known as \emph{functionals}.
The space of functionals intuitively seems to be much “bigger” than the space of functions, thus it should be a bit surprising that if $X$ is a Hilbert space, then $X^\ast$ is actually isomorphic to $X$ (Riesz representation theorem). Crucial for this is that $X^\ast$ includes only linear functions.

But actually, even simple-looking linear functionals do not typically correspond to functions in $X$. In particular, there are two conspicuous classes of linear functionals which cannot both be mapped to functions in a consistent way:
\begin{itemize}
 \item Evaluation at a single point (or a discrete collection of points). % Add example formula
 \item Integration over an interval.  % +example
\end{itemize}
The reason that this does not contradict the Riesz theorem is that the standard Hilbert spaces of functions actually contain \emph{equivalence classes} of functions, rather than individual functions. In the $\mathcal{L}^2$ space, functions which differ only in a single point are considered the same vector.

On one hand, this can be physically motivated (limited precision, so it would never be possible to focus an experiment on a single point anyway).
On the other hand, theoretical physics is actually thrive with calculations that do involve single-point evaluations. There they are called \emph{Dirac distribution}, often introduced informally as
\begin{align*}
  \delta &:\quad \mathbb{R} \to “\mathbb{R}\cup\{\infty\}”
 \\
  \delta(x) &= \begin{cases} “\infty” & \text{if }x=0
                          \\ 0        & \text{otherwise} \end{cases}.
\end{align*}
But the actual \emph{intention} behind this is simply the functional
\begin{align*}
  \delta^\ast &:\quad (\mathbb{R} \to \mathbb{R}) \to \mathbb{R}
 \\
  \delta^\ast(f) &= f(0).
\end{align*}
This is in physics written
\[
  \delta^\ast(f) = \int_\mathbb{R}\!\mathrm{d}x\: \delta(x)\cdot f(x),
\]
pretending that the Dirac distribution is actually a real function.


%% Acknowledgments
\begin{acks}                            %% acks environment is optional
                                        %% contents suppressed with 'anonymous'
  %% Commands \grantsponsor{<sponsorID>}{<name>}{<url>} and
  %% \grantnum[<url>]{<sponsorID>}{<number>} should be used to
  %% acknowledge financial support and will be used by metadata
  %% extraction tools.
  This material is based upon work supported by the
  \grantsponsor{GS100000001}{National Science
    Foundation}{http://dx.doi.org/10.13039/100000001} under Grant
  No.~\grantnum{GS100000001}{nnnnnnn} and Grant
  No.~\grantnum{GS100000001}{mmmmmmm}.  Any opinions, findings, and
  conclusions or recommendations expressed in this material are those
  of the author and do not necessarily reflect the views of the
  National Science Foundation.
\end{acks}


%% Bibliography
%\bibliography{bibfile}


%% Appendix
\appendix
\section{Appendix}

Text of appendix \ldots

\end{document}
